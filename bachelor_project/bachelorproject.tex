\documentclass[a4paper]{usiinfbachelorproject}

\captionsetup{labelfont={bf}}
%%%%%%%%%%%%%%%%%%%%%%%%%%%% PACKAGES %%%%%%%%%%%%%%%%%%%%%%%%%%%%%
\usepackage{float}
\usepackage{amsmath}
\usepackage{subfig}
\usepackage{graphicx}
\usepackage{svg}

%%% Main Body %%%

\author{Sasha Toscano}

\title{\textbf{Understanding arbitrary code
execution}}
\subtitle{A case study on Pokémon Emerald}
\versiondate{\today}

\begin{committee}
%With more than 1 advisor an error is raised...: only 1 advisor is allowed!
\advisor[Universit\`a della Svizzera Italiana, Switzerland]{ }{Carlo Alberto}{Furia}
%You can comment out  these lines if you don't have any assistant
\coadvisor[Universit\`a della Svizzera Italiana, Switzerland]{ }{Marc}{Langheinrich}

\end{committee}

\abstract { Abstract goes here ...
You may include up to six keywords or phrases. Keywords should be separated with semicolons. 
\\
\textbf{Keywords}:

}
\begin{document}
\maketitle
\tableofcontents\newpage
%\listoffigures\newpage

\section{Introduction}
In the world of software, security is one of the most critical concepts as it is the foundation of trust in the software we use every day. Let it be something as simple as our browser remembering our passwords, or as complex as a bank's online banking system, we need to be able to trust that the software we use is secure and that our data is safe and well protected. This is especially important in this day and age where we rely on these types of software for everything: from communication, to banking, and even to entertainment.

The software we use is often complex and interconnected, making it difficult to distinguish what is secure and what isn't. In many cases, these programs are often developed quickly and with limited resources, and as a result, security is often an afterthought in the development process, leading to oversights or the introduction of vulnerabilities that can be exploited by attackers.

One of the most common types\footnote{https://www.sciencedirect.com/topics/computer-science/arbitrary-code-execution} of these vulnerabilities is arbitrary code execution (ACE), which allows a malicious unauthorized user to execute some arbitrary code on a target system or in a specific application. This can happen because software or computer systems in general are not capable of differentiating between some generic text and actual commands, without some proper protections put in place.

This then becomes a significantly serious issue as it can lead to unauthorized access to sensitive information, data loss, and even complete system compromise. These types of ACE vulnerabilities can be found in a wide range of software: including operating systems, web applications, and even video games. In fact, some of the most interesting ACE vulnerabilities have been found in video games like \textit{Super Mario World}\footnote{https://arstechnica.com/gaming/2014/01/how-an-emulator-fueled-robot-reprogrammed-super-mario-world-on-the-fly/}, where they have been used to allow users to literally "program" new games into the system to be played.

In this thesis, we will explore the concept of ACE using a case study of the game \textit{Pokémon Emerald} (2004) to demonstrate how such vulnerabilities can be exploited. We will then examine the current state of ACE-related attacks and discuss modern techniques that can be used to mitigate them. We will also look at some of the most common techniques used to protect against these vulnerabilities, and how they can be prevented.

\section{Arbitrary code execution}
\label{sec:ace}
I'd like to start with a simple explanation of what arbitrary code execution is: \textit{updog}. \\ And what is updog, you may ask? \textit{Not much, what's up with you?}

\begin{figure}[h!]
	\center{\includegraphics[scale=0.3]
		{figures/ace.png}}
	\caption{A graphical illustration of the \textit{updog} joke\label{fig:updog}}
\end{figure}

This joke, illustrated in figure \ref{fig:updog} (credits to Smoovers\footnote{https://www.youtube.com/watch?v=-XmXYCXX7y4}) is an extremely simple but direct way of explaining what ACE is. In this case the joke is self-explanatory, but the idea behind it is that we end up saying something (\textit{"what's up dawg?"}) that is not what we intended to say (\textit{"what does updog mean?"}). This is extremely similar to what happens during ACE in the real world, where the system ends up executing code that we did not intend to execute, albeit with dramatically more complicated consequences.

To get a bit more into the proper definition of things: ACE is a type of flaw or vulnerability that allows an attacker to execute some generally speaking malicious (or at least unauthorized) code on a target system\footnote{https://www.okta.com/identity-101/arbitrary-code-execution/}. Without proper protections and preventions put in place, an attacker can easily exploit these to gain access to sensitive information, take control of the system, or even cause damage to the system itself. The severity of this incidents can vary greatly: in the past these exploits have gone from something low-level like allowing gamers to better their speedrunning performances\footnote{https://www.youtube.com/watch?v=A2DlyLPr4Ws} (where speedrunning is the art of playing a video game with the goal of completing it as fast as possible), to something a fair bit more complicated, like leaking sensitive kernel memory on real world systems, as was the case in Retbleed\footnote{https://comsec.ethz.ch/research/microarch/retbleed/}.

These vulnerabilities can be exploited through various means, including buffer overflows, code injection, and other techniques; however in the context of this case study, we will be looking at a specific example of ACE in the world of video games, specifically in the game \textit{Pokémon Emerald} where the techniques used to exploit the vulnerabilities are \textbf{arbitrary memory write} and \textbf{out-of-bounds writes}, which is where existing mechanics are used to place crafted instructions into writable areas of memory.




\section{Pokémon Emerald: A Case Study of Arbitrary Code Execution}
\label{sec:case_study}
In this section, we explore how players can manipulate the inner workings of \textit{Pokémon Emerald} to execute their own code within the game, a phenomenon that occurs through ACE. This process is made possible by chaining together several in-game glitches that interact in unintended ways as seen in figure \ref{fig:ace_scheme}, ultimately allowing the player to influence memory and control the game’s behavior.

\begin{figure}[h!]
	\center{\includegraphics[width=0.5\textwidth]
		{figures/ace_img.png}}
	\caption{A scheme illustrating the process to achieve ACE}\label{fig:ace_scheme}
\end{figure}

The starting point for this chain is the \textit{Pomeg glitch} (chapter~\ref{sec:pomeg_glitch}), a glitch that allows the player to reduce a Pokémon’s HP below zero outside of battle, causing the game to enter an invalid state. When attempting to initiate a battle under these conditions, the game creates a placeholder Pokémon to avoid crashing, unknowingly opening the door to deeper exploitation. By examining this placeholder in battle, players can set up conditions to trigger another glitch known as \textit{Glitzer Popping} (chapter~\ref{sec:glitzer_popping}), which involves renaming the in-game PC storage boxes in a precise way. This process, along other steps, ultimately leads the game into treating the renamed data as if it were valid Pokémon data, and eventually as executable code.
This setup causes memory to be misread and reinterpreted and with careful planning, players can direct the game to areas of memory that contain useful data, or even custom payloads, by taking advantage of how the game processes corrupted or invalid Pokémon entries. While this avoids the need for external tools or hacking devices, rendering this types of manipulations doable on any type of legitimate hardware (like a Game Boy), the level of control granted is surprisingly deep, even able to altering game logic and spawning items or Pokémon at will.
This can become a serious issue, as, although technically impressive and often used only for speedrunning, this type of ACE could have lead to a monetary loss for the Pokémon Company. This is because, when the game originally came out in 2004, the company used to hold special in-person events where some rare Pokémon were given out to players (and this used to be the only way to get these Pokémon), as such, for anyone to be able to "generate" them through a couple of glitches it would make this a problem for the company, as the players would not need to attend this highly publicized events to get the Pokémon.


\subsection{The Pomeg glitch}
\label{sec:pomeg_glitch}
In Pokémon Emerald, the player can use a glitch called \textit{Pomeg glitch} to put the game in an impossible state. To explain this glitch first it would be wise to understand how the game roughly works. In Pokémon Emerald the player can be in two states: the exploration state, where they can go around the map with a team of up to six Pokémon, and the battle state, which is where the fighting happen. Any Pokémon caught after the sixth gets sent in the PC, which is a storage system that allows the player to store Pokémon that they do not want to carry with them. The Pokémon have their own statistics, which improve based on their levels and what other Pokémon they fight. The statistic involved with the Pomeg glitch is \textit{HP} (Hit Points), which is the amount of health a Pokémon has. If a Pokémon's HP reaches 0, it faints and cannot be used until it is revived plus, if all of them reach 0, we get a game over and this can only happen when the game is in the battle state.

The HP statistic is calculated through the following formula:

\begin{equation}
	\text{HP} = \left( \frac{(2 \times \text{Base} + \text{IV} + \left( \frac{\text{EV}}{4} \right)) \times \text{Level}}{100} \right) + \text{Level} + 10
	\label{eq:pokemon_hp_formula}
\end{equation}

This formula yields the maximum HP of a Pokémon, which is the maximum amount of health it can have. In the case of the \textit{Pomeg glitch}, the only relevant variable is the EV value: this is because the EV value is the only statistic that can be directly modified based on the actions of the player. It is an integer between 0 and 255, and it increases every time a Pokémon defeats another, but most importantly it can also be decreased by using certain items. In the case of the \textit{Pomeg glitch}, the player can use a specific item called \textit{Pomeg berry} (hence the name) to lower the EV value of a Pokémon by 10. However, if, for example, the Pokémon's current HP was at 1, when updating the newly calculated HP statistic, both the current HP and max HP get lowered and as such, the game could set the current HP value to 0, or, even worse, something below it, like underflowing to $ 2^{16}-1$ or 65535 HP (due to the HP statistic being an unsigned two-bytes integer). Now, to use a \textit{Pomeg berry} the player needs to not be in the combat state, and this can create some issues because the game is only capable of handling game overs (which happens when all of the player's Pokémon reach 0 HP) if the player is in a combat state. The problem arises in the situation where the Pokémon whose HP is being updated is the only one with $HP > 1$, because the player may end up with a party of only dead Pokémon, as can be seen in the figure below.

\begin{figure}[h!]
	\center{\includegraphics
		{figures/dead_team.png}}
	\caption{A technically impossible state}\label{fig:dead_team}
\end{figure}

This is not a problem right as it happens, however if the player is to start a battle, the game will try to look for a Pokémon to fight with, and it will find none, because all of them are dead. So after unsuccessfully looking for a Pokémon, the game will then send out a \textbf{?-like} Pokémon (commonly known in the Pokémon community as \textbf{Decamark})\footnote{https://bulbapedia.bulbagarden.net/wiki/Ten\_question\_marks} as can be seen in figure \ref{fig:decamark}, which is used to prevent game crashes, as when the game fails to load a proper Pokémon sprite, the Decamark is used as a placeholder.

\begin{figure}[h!]
	\center{\includegraphics
		{figures/decamark.png}}
	\caption{The decamark Pokémon}\label{fig:decamark}
\end{figure}

At this point, the player can try to fight, run away or use an item, all options which would result in a white out (the game over mechanic). But how can this lead to ACE? This is where the glitch known as \textit{Glitzer Popping} comes into play.

\subsection{Glitzer Popping}
\label{sec:glitzer_popping}
The \textit{Pomeg data corruption glitch}, most commonly known as the \textit{Glitzer Popping}, is a glitch that allows the player to execute some arbitrary code by renaming the boxes to a specific string, which will then be interpreted as some ARM-Assembly code (since Pokémon Emerald is written in C and ARM-Assembly\footnote{https://pokemondb.net/pokebase/257153/what-programming-language-generations-series-games-written}) that will be executed when the game tries to read the name of the box. The player can rename the boxes in many different ways, obtaining different results spanning from obtaining event-exclusive items or Pokémon (for Pokémon Emerald many events were held in-person at some specific locations), to modify player information or game flags, all the way to being able to execute custom scripts (like creating your own events). "But how do we get the game to read the name of the box to the point where it gets executed?" is the most logical follow-up question. The answer is simple: we need to use the \textit{Pomeg glitch}.


\subsubsection{The setup}
\label{sec:setup}
First things first, this thesis is not a tutorial on how to achieve these glitches, as there are countless fantastic other guides about replicating them (like this video made by Youtuber PapaJefé\footnote{https://www.youtube.com/watch?v=45kwOEhKbUg}, or this guide made by Mickaël Laurent\footnote{https://e-sh4rk.github.io/ACE3/}), rather, my purpose is to illustrate and figure out why and how does ACE happen, using Pokémon Emerald as a basis for it.

To actually start executing some arbitrary code we need two things: the first one is the previously illustrated \textit{Pomeg glitch}, where the player is able to setup an impossible game state, and the second one needs the boxes in which the Pokémon that don't fit in the player's team are stored. These are 14 different boxes in which the Pokémon are stored, and they can be renamed by the player. The names of these boxes are stored in a specific area of the memory, and this is where \textit{Glitzer Popping} comes into play: as previously stated it is a technique that allows the player to execute some arbitrary code by renaming the boxes to a specific string, which will then be interpreted as some code that will be executed when the game tries to read the name of the box (an example can be seen in table \ref{tab:boxes_names}; \textbf{note}: in the game, the three dots are represented by a single character "\dots").

\begin{table}[h!]
	\centering
	\begin{tabular}{|c|l|}
		\hline
		\textbf{Box} & \textbf{Name}  \\
		\hline
		Box 1        & (VTTnFMBn)     \\
		Box 2        & (EEENJRo )     \\
		Box 3        & (EE5\dots q  ) \\
		Box 4        & (EFF\dots    ) \\
		Box 5        & (KT?nTR?n)     \\
		Box 6        & (EEEIP?n )     \\
		Box 7        & (EE'FQm  )     \\
		Box 8        & (EmFlo   )     \\
		Box 9        & (yLRom"Ro)     \\
		Box 10       & (EEEFGEn )     \\
		Box 11       & (EE \dots q  ) \\
		Box 12       & (E\dots -n   ) \\
		Box 13       & (FQRn\dots Rn) \\
		Box 14       & (EEEt ?n )     \\
		\hline
	\end{tabular}
	\caption{Example box names required to teleport to map ID \texttt{0B10}}
	\label{tab:boxes_names}
\end{table}


The game is not capable of handling the situation where the player has a Decamark in their team, and as such when the player tries to open their team information in battle it will unsuccessfully try to read the Pokémon's data from the memory, leaving instead a blank Pokémon.

So when the player moves the cursor to the first slot in their team, the cursor actually underflows and goes to slot 256 (since a team is made up of a maximum of 6 Pokémon) and as such, instead of scrolling through slots 1-6, the player has access to slots 255 and above (as the cursor is technically out of bounds). However, slots outside the first 6 aren't meant to be Pokémon slots, so the game accesses random blocks of RAM data and treats them as Pokémon, even though they are not. In this case the 255th party slot is actually the PC Pokémon data and continuing to scroll upwards allows the player to actually go over memory addresses reserved for various in-game data.

Each time the party Pokémon selection pointer moves to a new party slot, an anti-cheat verification routine is triggered for the selected "Pokémon" (because the game thinks it is one, but it's actually not). If the checksum of the selected data block (interpreted as a Pokémon) is invalid, the game modifies it into a Bad Egg, which is the result of the anti-cheating protocol. This transformation involves setting the Egg Status flag to 1 and enabling two additional bits, which designate the egg as a "Bad" Egg, whose in-game aspect can be seen in figure \ref{fig:bad_egg}. Since the memory blocks being interpreted as party Pokémon are not genuine Pokémon structures, their checksums will almost always be invalid unless the slot is empty, or accurately prepared.

\begin{figure}[h!]
	\center{\includegraphics
		{figures/bad_egg.png}}
	\caption{A corrupted Pokémon that got labeled as a "Bad Egg"}\label{fig:bad_egg}
\end{figure}

\subsubsection{Pokémon data structure}
\label{sec:data_structure}
A Pokémon's data structure is made up of 100 bytes (which can be seen in table \ref{fig:full-pokemon-structure}), with many static (in the sense that once created they never get modified again) and some other dynamic fields (in the sense that they can be modified during the game). The static fields are used to store information about the Pokémon, such as its species, nickname, and trainer ID (TID, which is a 4-bytes (a double-word in C) long integer ranging from 0 to 65'535 that gets generated when the player starts the game for the first time), while the dynamic fields are used to store information about the Pokémon's current state, such as its level, experience points, current HP, etc.
The order of these information remains the same for every Pokémon except for the "Inner data" substructure (the substructure that starts at offset 0x20 in table \ref{fig:full-pokemon-structure}) which contains other generic information about the Pokémon: the structure is divided into four substructures, each of which is 12 bytes long. The first substructure contains the growth information, the second one contains the attacks, the third one contains the EVs and condition, and the last one contains miscellaneous information.

\begin{table}[h!]
	\centering
	\begin{tabular}{|c|c|c|}
		\hline
		\textbf{Name}          & \textbf{Offset (hex)} & \textbf{Length (bytes)} \\
		\hline
		Personality value      & 0x00                  & 4                       \\
		OT ID                  & 0x04                  & 4                       \\
		Nickname               & 0x08                  & 10                      \\
		Language               & 0x12                  & 1                       \\
		Misc. Flags            & 0x13                  & 1                       \\
		OT name                & 0x14                  & 7                       \\
		Markings               & 0x1B                  & 1                       \\
		Checksum               & 0x1C                  & 2                       \\
		Padding                & 0x1E                  & 2                       \\
		Inner data (encrypted) & 0x20                  & 48                      \\
		Status condition       & 0x50                  & 4                       \\
		Level                  & 0x54                  & 1                       \\
		Mail ID                & 0x55                  & 1                       \\
		Current HP             & 0x56                  & 2                       \\
		Total HP               & 0x58                  & 2                       \\
		Attack                 & 0x5A                  & 2                       \\
		Defense                & 0x5C                  & 2                       \\
		Speed                  & 0x5E                  & 2                       \\
		Sp. Attack             & 0x60                  & 2                       \\
		Sp. Defense            & 0x62                  & 2                       \\
		\hline
	\end{tabular}
	\caption{Full Pokémon Data Structure (Generation III)}
	\label{fig:full-pokemon-structure}
\end{table}

This substructure is the most relevant for a Pokémon as it contains the most important information about the Pokémon itself, on top of being key to verify a Pokémon's data integrity. How that part is handled is that the order of these four substructures is determined by the Pokémon's Personality Value (PID, which is an 8-bytes long integer that gets generated when a Pokémon's data is first created) and the TID. The formula used to determine the order of the substructures is as follows:

\begin{equation}
	\left( \texttt{TID} \oplus \texttt{PID} \right) \bmod 24
	\label{eq:substructure_order_formula}
\end{equation}

The game takes the TID and the PID, lengthens the TID to the length of the PID by adding 0s to it and XORs it with the PID, and then takes the result modulo 24. The result of this calculation is then mapped directly to the substructure order table (table \ref{tab:substructure-order}) to determine the order of the substructures, where G stands for the growth substructure, A stands for the attacks one, E is the EVs and condition one and M is the miscellaneous substructure.

\begin{table}[h!]
	\centering
	\begin{tabular}{cccc}
		\toprule
		00. GAEM & 06. AGEM & 12. EGAM & 18. MGAE \\
		01. GAME & 07. AGME & 13. EGMA & 19. MGEA \\
		02. GEAM & 08. AEGM & 14. EAGM & 20. MAGE \\
		03. GEMA & 09. AEMG & 15. EAMG & 21. MAEG \\
		04. GMAE & 10. AMGE & 16. EMGA & 22. MEGA \\
		05. GMEA & 11. AMEG & 17. EMAG & 23. MEAG \\
		\bottomrule
	\end{tabular}
	\caption{Substructure order based on PID\%24}
	\label{tab:substructure-order}
\end{table}


Because these substructures are encoded using the Pokémon’s PID and the Trainer ID (TID), setting the Egg Status flag to 1 can result in either a bit value of 1 or 0, depending on the result of \texttt{PID} XOR \texttt{TID}. Out of all of the different locations within a Pokémon's data structure, the Egg Status flag resides in the miscellaneous slot, however, in contrast, the two bits that define a "Bad" Egg are always located at fixed positions at offset 0x13 of the Pokémon's data in the "Misc. flags" slot, and they are consistently set to 1 when a checksum is invalid.

\begin{table}
	\begin{center}
		\begin{tabular}{|c|c||c|c||c|c||c|c|}
			\hline
			\multicolumn{2}{|c||}{\textbf{Growth}}          &
			\multicolumn{2}{c||}{\textbf{Attacks}}          &
			\multicolumn{2}{c||}{\textbf{EVs \& Condition}} &
			\multicolumn{2}{c|}{\textbf{Miscellaneous}}                                                                              \\
			\hline
			\textbf{Name}                                   & \textbf{Bytes} &
			\textbf{Name}                                   & \textbf{Bytes} &
			\textbf{Name}                                   & \textbf{Bytes} &
			\textbf{Name}                                   & \textbf{Bytes}                                                         \\
			\hline
			Species                                         & 2              & Move 1 & 2 & HP EV      & 1 & Pokérus status      & 1 \\
			Item held                                       & 2              & Move 2 & 2 & Attack EV  & 1 & Met location        & 1 \\
			Experience                                      & 4              & Move 3 & 2 & Defense EV & 1 & Origins info        & 2 \\
			PP bonuses                                      & 1              & Move 4 & 2 & Speed EV   & 1 & IVs, Egg, Ability   & 4 \\
			Friendship                                      & 1              & PP 1   & 1 & Sp. Atk EV & 1 & Ribbons + Obedience & 4 \\
			Unused                                          & 2              & PP 2   & 1 & Sp. Def EV & 1 &                     &   \\
			                                                &                & PP 3   & 1 & Coolness   & 1 &                     &   \\
			                                                &                & PP 4   & 1 & Beauty     & 1 &                     &   \\
			                                                &                &        &   & Cuteness   & 1 &                     &   \\
			                                                &                &        &   & Smartness  & 1 &                     &   \\
			                                                &                &        &   & Toughness  & 1 &                     &   \\
			                                                &                &        &   & Feel       & 1 &                     &   \\
			\hline
		\end{tabular}
		\caption{Pokémon data structure by substructure (12-bytes blocks)}
		\label{table:pokemon-data-structure}
	\end{center}
\end{table}

\newpage

An additional layer of randomness is introduced by the DMA (Direct Memory Access) system, which is another built-in anti-cheat mechanism that shifts the RAM addresses of numerous data structures whenever the player performs actions such as entering a battle, going through a doorway, or opening the in-game bag. The DMA remaps memory addresses using a translation table of multiple double-words. Any value subjected to DMA can occupy up to 32 different memory addresses, each spaced by 4 bytes.

Importantly, party Pokémon are not affected by DMA, which means the memory addresses of the six standard party slots remain fixed. However, any data read from memory locations beyond the sixth slot \textit{is} subject to DMA. Since each party Pokémon occupies 25 double-words, and DMA remapping allows up to 32 double-word shifts, each double-word in a slot beyond the sixth could potentially be placed at an address susceptible to corruption via the Egg Status bit alterations. Due to the variability in both RAM content and corruption locations, these elements may interact unpredictably, sometimes preventing a given double-word from being corrupted by the Egg-related flags.

Through the use of carefully planned strategies (like the \textit{Glitzer Popping}), it becomes possible to intentionally corrupt specific values within the data of a Pokémon, while simultaneously ensuring that surrounding data remains unaffected, thus enabling a form of highly targeted or pinpoint corruption. This capability allows for the precise manipulation of a Pokémon's PID and/or TID within the PC storage system, without altering the remainder of the Pokémon’s structured data. Since both the PID and TID serve a crucial role in encrypting Pokémon’s four internal substructures, with each one containing different categories of the Pokémon’s information, the corruption of these values causes a dramatic shift in the Pokémon’s checksum, which functions as a form of data integrity verification.

Among the known corruption methods, the two specific bits responsible for creating so-called "Bad Eggs" are not suitable for intentional and controlled Pokémon data corruption, as they invariably fail to preserve the checksum, rendering the resulting data unstable and largely unusable. In contrast, a more nuanced method (corruption via the Egg State Flag) offers a much more promising avenue, as it allows for manipulation while keeping the checksum intact. This method alters the checksum by a multiple of 0x4000, and because the checksum is stored as a 16-bit word, any change that corresponds to an even multiple leaves the checksum effectively unchanged. Although there exist certain rare conditions under which this multiple might be odd, thus compromising the checksum, these conditions can be easily identified and prevented, allowing for a consistently reliable corruption method that keeps the data within safe bounds.

Since the PID governs the order in which the four data substructures are stored and subsequently read by the game, corrupting it directly alters this order, resulting in the game interpreting one category of data as another: for example, reading what is meant to be the Moves substructure as if it were the EVs substructure. This misinterpretation of substructure order becomes a powerful tool for altering multiple facets of a Pokémon's attributes such as its species, held item, experience points, individual values (IVs), effort values (EVs), friendship level, obedience flag, origin data, and of course, its moveset, simply by assigning specific values to those fields prior to the corruption process. Although there are theoretically 24 different ways to reorder the four substructures, only 10 permutations are actually possible within the game’s mechanics when doing a corruption. These ten permutations are collectively referred to as Corruption Types, and each one produces unique effects based on how the substructure data is reorganized, meaning that the specific impact of a PID corruption is entirely determined by which Corruption Type is applied.

However, even when using the Egg State Flag corruption technique, which successfully preserves the Pokémon’s checksum and alters the substructure order as intended, the change to the encryption key (caused by modifying either the PID or TID) can still introduce undesirable side effects. These side effects include transforming the Pokémon into an Egg, assigning glitched or unusable moves to the second and fourth move slots, or disrupting other internal values that affect gameplay. The presence of a corrupted Pokémon in an Egg state poses a significant drawback, as the act of hatching resets many of its attributes and, in the case of certain Glitch Pokémon, can result in game crashes or freezing. Furthermore, corrupted move slots can restrict the player from using, viewing, reordering, or replacing the affected moves, thus limiting the Pokémon's functionality and strategic use.

To mitigate these problems and produce a clean, stable result, an advanced technique involves performing two successive \textit{Glitzer Popping} procedures, first corrupting the PID, and then the TID (or vice versa) in a way that ensures the Pokémon ends up with a valid checksum, a desired change in substructure order, and, importantly, a restoration of its original encryption key through the reversal of the PID or TID alteration. This dual-step approach ensures that the corrupted Pokémon retains its intended modifications while eliminating residual glitched values, effectively enabling the precise and consistent corruption of nearly any Pokémon without the drawbacks typically associated with such manipulations.


\subsubsection{Running the code}
\label{sec:running_the_code}
Through the aforementioned glitches, it's possible to create a corrupted Pokémon that, when hatched, due to the stable corruption will have some of its data switched: due to the corruption of the PID and the TID, now the result of equation \ref{eq:substructure_order_formula} will be different, and as such the order of the substructures will be different, but it will maintain the previous values. For example a Pokémon whose checksum gave it an order of GAEM could (since the corruption has some casualty to it due to DMA) end up being ordered as EAGM, which means that the game will read the data in a different order than it was originally intended and if the HP EV value was 0x06 (6 in decimal) while the Attack EV value was 0x11 (17 in decimal), the game will now read the Species value as 0x0611 (since the Species value uses 2 bytes and the HP and Attack EVs only use 1) and vice-versa (this process can be seen visualised in figure \ref{fig:corruption_substructures}). This, as previously mentioned, creates a problem when trying to reference the Pokémon's sprite animation, as the address of one's sprite animation is directly based on its species value.

\begin{figure}[h!]
	\center{\includegraphics[width=0.85\textwidth]
		{figures/corruption_substructures.png}}
	\caption{Visual representation of the corruption of the substructures}
	\label{fig:corruption_substructures}
\end{figure}

Therefore, when a Pokémon egg hatches, the animation gets played (as shown in figure \ref{fig:sprite_animation}), and the game will then try to read the Pokémon data from the memory, however the address is not the one of a valid animation since in Pokémon Emerald only IDs of up to 0x0183 are occupied by actual animations.

\begin{figure}[!htb]
	\centering
	\subfloat[First frame]{
		\includegraphics[width=0.25\textwidth]{figures/charmander_hatch.png}}
	\label{fig:First frame of the hatching animation}
	\qquad
	\subfloat[Second frame]{
		\includegraphics[width=0.25\textwidth]{figures/char_animation.png}}
	\label{fig:Second frame of the hatching animation}
	\qquad
	\subfloat[Final frame]{
		\includegraphics[width=0.25\textwidth]{figures/char_end_animation.png}}
	\label{fig:Final frame of the hatching animation}
	\caption{Sprite animations}
	\label{fig:sprite_animation}
\end{figure}


As seen in table \ref{tab:boxes_names_explanation}, the code that gets executed ultimately ends up actually being a series of ARM Assembly instructions, which are then executed by the game. And this is ultimately where ACE happens: since ARM Assembly is a low-level programming language, if the species ID points to a "wrong" or "unintended" entity, instead of crashing, it just reads and executes what is found there. In our example the table for the animations has valid entries up to 0x50, but since the function that handles this is capable of reading up to 0xFF, and entry 0x6e is where the PC boxes names start, the game erroneously reads and executes the PC box names.

\begin{table}
	\begin{tabular}{|l|l|l|p{6cm}|}
		\hline
		\textbf{Instruction}                & \textbf{Opcode (Hex)} & \textbf{Character Used} & \textbf{Description}               \\
		\hline
		\texttt{sbc r12, lr, \#0xD00}       & E2CECED0              & --                      & R12 = LR - 0xD00 - 1               \\
		\texttt{adcs r12, r12, \#0x3000000} & E2BCC7C0              & --                      & R12 += 0x3000000 (offset setup)    \\
		\texttt{-filler-}                   & BFBFBFFF              & --                      & Padding                            \\
		\texttt{bic r12, r12, \#0xC8000000} & E3CCC4C8              & --                      & Clear upper bits for alignment     \\
		\texttt{-filler-}                   & BFBFFF00              & --                      & Padding                            \\
		\texttt{ldrt r11, [r12, \#0xA6]!}   & E5BCB0A6              & --                      & Load map ID address into R11       \\
		\texttt{-filler-}                   & BFFF0000              & --                      & Padding                            \\
		\texttt{mov r12, \#'F'}             & E3B0C04F              & 'F'                     & From Box 4 → Load base character   \\
		\texttt{-filler-}                   & FF000000              & --                      & Padding                            \\
		\texttt{adc r12, r12, \#'K'}        & E2ACCE4B              & 'K'                     & From Box 5 → Add next character    \\
		\texttt{adc r12, r12, \#'T'}        & E2ACCC54              & 'T'                     & From Box 5 → Add next character    \\
		\texttt{-filler-}                   & BFBFBFFF              & --                      & Padding                            \\
		\texttt{adc r12, r12, \#'I'}        & E2ACC049              & 'I'                     & From Box 6 → Final character added \\
		\texttt{-filler-}                   & BFBFFF00              & --                      & Padding                            \\
		\texttt{strh r12, [r11, \#4]}       & E1CBC0B4              & FKTI                    & Store constructed map ID           \\
		\texttt{-filler-}                   & BFFF0000              & --                      & Padding                            \\
		\texttt{mvn r12, \#0xE1}            & E3E0C0E1              & --                      & Prepare to build opcode            \\
		\texttt{-filler-}                   & FF000000              & --                      & Padding                            \\
		\texttt{bic r12, r12, \#0xED00000}  & E3CCC6ED              & --                      & Clear bits                         \\
		\texttt{bic r11, r12, \#0x1000000E} & E3CCB2E1              & --                      & Prepares opcode for `bx r0`        \\
		\texttt{-filler-}                   & BFBFBFFF              & --                      & Padding                            \\
		\texttt{adcs r12, pc, \#0x30}       & E2BFC1C0              & --                      & Compute destination address        \\
		\texttt{-filler-}                   & BFBFFF00              & --                      & Padding                            \\
		\texttt{strt r11, [r12]!}           & E5ACB000              & --                      & Store `bx r0` instruction          \\
		\texttt{-filler-}                   & BFFF0000              & --                      & Padding                            \\
		\texttt{adc r12, lr, \#0xB0000}     & E2AECAB0              & --                      & Adjust base pointer                \\
		\texttt{-filler-}                   & FF000000              & --                      & Padding                            \\
		\texttt{sbc r12, r12, \#0x30000}    & E2CCCBC0              & --                      & Subtract offset                    \\
		\texttt{sbc r12, r12, \#0xB00}      & E2CCCEB0              & --                      & More alignment                     \\
		\texttt{-filler-}                   & BFBFBFFF              & --                      & Padding                            \\
		\texttt{adc r0, r12, \#0xE8}        & E2AC00E8              & --                      & Final call to `CB2\_LoadMap2ENG`   \\
		\hline
	\end{tabular}

	\caption{Explanation of the commands seen in the previous table}
	\label{tab:boxes_names_explanation}
\end{table}



\section{Detecting and Preventing Arbitrary Code Execution}
\label{sec:state_of_the_art}

In the previous chapter \ref{sec:case_study} we have seen how ACE can be achieved through the use of glitches in Pokémon Emerald, and how it is possible to execute some arbitrary code. In figure \ref{fig:ace_scheme} we saw how ACE is achievable through the chaining of glitches, and the first one of these, the \textit{Pomeg glitch} is born out of a simple missing if statement that doesn't check if the Pokémon's HP is greater than 1 before trying to use the berry. This is a simple mistake that can be easily fixed nowadays, either through updates or patches, but in 2004 when the game originally released this was not intended. In this chapter we will see how ACE can be detected and prevented in modern systems, looking at the current state of the art in the field of ACE detection and prevention analysing much recent scenarios and papers.
Initially, in chapter \ref{sec:today} the focus will be on detection, and the article we will be looking at is \textit{"Detection of Remote Code Execution vulnerability in website source codes using LSTM machine learning model"}\footnote{https://www.researchgate.net/publication/381650864\_Detection\_of\_Remote\_Code\_Execution\_vulnerability\_in\_website\_source\_codes\_using\_LSTM\_machine\_learning\_model} by Armin Zakarian and Muhammad Rahmani, which was published in 2024. The paper focuses on the detection of Remote Code Execution (RCE) vulnerabilities in web applications, which is a type of ACE that can be exploited by attackers to execute arbitrary code on a target system.
The authors propose a machine learning-based approach to detect RCE vulnerabilities in web application source codes, using Long Short-Term Memory (LSTM) networks to analyze the source code and identify potential vulnerabilities.
After which, the focus, in chapter \ref{sec:current_studies}, will be on prevention, and the article we will be looking at is \textit{"EC-CFI: Control-Flow Integrity via Code Encryption Counteracting Fault Attacks"}\footnote{https://arxiv.org/abs/2301.13760} which was published in 2023. The paper focuses on the prevention of ACE vulnerabilities in embedded systems, which are often targeted by attackers due to their limited resources and lack of security measures.


\subsection{Detection}
\label{sec:today}

\subsection{Prevention}
\label{sec:current_studies}







\newpage

\section{Summary/Conclusions}






\newpage

%%%%% BIBLIOGRAPHY %%%%%
\bibliographystyle{abbrv}
\bibliography{references}
\begin{itemize}
	\item \url{https://www.sciencedirect.com/science/article/abs/pii/B9781597496537000104}
	\item \url{https://thesecmaster.com/blog/how-to-fix-the-cve-2021-40444}
	\item \url{https://www.okta.com/identity-101/arbitrary-code-execution/}
	\item \url{https://catonmat.net/ldd-arbitrary-code-execution}
	\item \url{https://www.wallarm.com/what/arbitrary-code-execution-vulnerabilities}
	\item \url{https://comsec.ethz.ch/research/microarch/retbleed/}
	\item \url{https://azeria-labs.com/writing-arm-assembly-part-1/}
	\item \url{https://arxiv.org/abs/2301.13760}
	\item \url{https://dl.acm.org/doi/10.1145/1920261.1920269}
	\item \url{https://glitchcity.wiki/wiki/Pomeg_glitch}
	\item \url{https://glitchcity.wiki/wiki/Pomeg_data_corruption_glitch}
	\item \url{https://projectpokemon.org/home/docs/other/notable-breakpoints-r31/}
	\item \url{https://bulbapedia.bulbagarden.net/wiki/Pomeg_glitch}
	\item \url{https://glitchcity.wiki/wiki/Arbitrary_code_execution}
	\item \url{https://problemkaputt.de/gbatek.htm#gbamemorymap}
	\item \url{https://tasvideos.org/6616S#GlitzerPopping}
	\item \url{https://bulbapedia.bulbagarden.net/wiki/Pok\%C3\%A9mon_data_substructures_(Generation\_III)}
\end{itemize}


\end{document}
